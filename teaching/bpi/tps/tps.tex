\documentclass{book}
\usepackage{color}
\usepackage{listings}
\usepackage[T1]{fontenc}
\ifdefined\HCode
\def\pgfsysdriver{pgfsys-tex4ht.def}
\fi
\usepackage{tikz}
\usetikzlibrary{calc}

\usepackage{graphicx}
\usepackage[french,linesnumbered]{algorithm2e}
\usepackage[french]{babel}
\usepackage{hyperref}
\usepackage{soul}

\lstset{
    basicstyle=\ttfamily,
    language=python,
    %numbers=left,
    keywordstyle=\rmfamily\bfseries,
    commentstyle=\sffamily,
}

\newcommand{\video}[1]{
    #1
}

\title{Bases de la programmation imp\'erative\\
Exercices de travaux pratiques}
\author{}
\date{}

\begin{document}
\maketitle

\Css{div.lstlisting .ectt-1000 {font-family: monospace;color:blue}}
\Css{div.lstlisting .ecss-1000 {font-family: monospace;color:green}}
\Css{div.lstlisting .ecbx-1000 {font-family: monospace;color:red}}
\Css{div.lstinputlisting .ectt-1000 {font-family: monospace;color:blue}}
\Css{div.lstinputlisting .ecss-1000 {font-family: monospace;color:green}}
\Css{div.lstinputlisting .ecbx-1000 {font-family: monospace;color:red}}
\input{1/bases.tex}
\input{2/iterations.tex}
\input{3/tableaux.tex}
\input{4/listes.tex}
\input{5/recursion.tex}

\end{document}
